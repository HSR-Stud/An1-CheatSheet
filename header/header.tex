%header 2

% Analysis 2 Formelsammlung
% HSR Elektrotechnik, 2. Semester

% Dokumenteinstellungen
% ======================================================================

% Dokumentklasse (Schriftgröße 6, DIN A4, Artikel)
\documentclass[6pt,a4paper,fleqn]{scrartcl}


% Pakete laden
\usepackage[utf8]{inputenc}		% Zeichenkodierung: UTF-8 (für Umlaute)
\usepackage[T1]{fontenc}		%Umlaute
\usepackage[ngerman]{babel}		% Deutsche Sprache
\usepackage{multicol}			% Spaltenpaket
\usepackage{amsmath}
\usepackage{color}
\usepackage{amssymb}
\usepackage{extarrows}
\usepackage{esint}				% erweiterte Integralsymbole
\usepackage{wasysym}			% Blitz
\usepackage{graphicx}

\usepackage{tabulary}
\usepackage{tabularx}			%Modifizierbare Spaltenbreite bei Tabellen


\usepackage{tikz}
\usetikzlibrary{datavisualization}
\usetikzlibrary{datavisualization.formats.functions}
\usepackage{pgfplots}



% Seitenlayout und Ränder:
\usepackage{geometry}
\geometry{a4paper,landscape, left=6mm,right=6mm, top=0mm, bottom=3mm,includeheadfoot}
\setlength{\mathindent}{0pt}	%Damit die Gleichungen linksbündig sind

%Kopf- und Fußzeile
\usepackage{fancyhdr}
\pagestyle{fancy}
\fancyhf{}

%\renewcommand{\footnotesize}{\textit}
\fancyfoot[L]{\footnotesize{\textbf{\titleinfo}}}
\fancyfoot[C]{\footnotesize{\authorinfo \hspace{5cm}  \licence}}
\fancyfoot[R]{\footnotesize{Stand: \today}}
\renewcommand{\headrulewidth}{0.0pt} %obere Linie ausblenden
\renewcommand{\footrulewidth}{0.4pt} %untere Linie


%Strich zwischen Seiten   
\setlength{\columnseprule}{0.4pt}

% Schriftart SANS für bessere Lesbarkeit bei kleiner Schrift
\renewcommand{\familydefault}{\sfdefault}

%Zeilenabstand
\usepackage{titlesec}

\titlespacing*{\section}
{0pt}{1ex plus 1ex minus .1ex}{1ex minus .1ex}
%{0pt}{2ex plus 1ex minus .2ex}{1ex}
\titlespacing*{\subsection}
{0pt}{1ex plus 1ex minus .1ex}{1ex minus .1ex}
\titlespacing*{\subsubsection}
{0pt}{1ex plus 1ex minus .1ex}{1ex minus .1ex}

\setlength{\parindent}{0em}


% Custom Commands
\renewcommand{\thesubsection}{\arabic{subsection}}
\newcommand{\me}[1]{\ensuremath{\left\{#1\right\}}}
\newcommand{\dme}[2]{\ensuremath{\left\{#1\,\vert\,#2 \right\}}}
\newcommand{\abs}[1]{\ensuremath{\left\vert#1\right\vert}}
\newcommand{\un}[1]{\; \unit{#1} }
\newcommand{\unf}[2]{\;\left[ \unitfrac{#1}{#2} \right]}
\newcommand{\norm}[2][\relax]{\ifx#1\relax \ensuremath{\left\Vert#2\right\Vert}\else \ensuremath{\left\Vert#2\right\Vert_{#1}}\fi}
\newcommand{\enbrace}[1]{\ensuremath{\left(#1\right)}}
\newcommand{\nira}[1]{\ensuremath{\overset{n \rightarrow \infty}{\longrightarrow}}}
\newcommand{\os}[2]{\ensuremath{\overset{#1}{#2}}}
\makeatletter
\newcommand{\Ra}[0]{\ensuremath{\Rightarrow}}
\newcommand{\ra}[0]{\ensuremath{\rightarrow}}
\newcommand{\gk}[1]{\ensuremath{\left\lfloor#1\right\rfloor}}
\newcommand{\sprod}[2]{\ensuremath{%
		\setbox0=\hbox{\ensuremath{#2}}
		\dimen@\ht0
		\advance\dimen@ by \dp0
		\left\langle #1\rule[-\dp0]{0pt}{\dimen@},#2\right\rangle}}

\newcommand{\tab}[1][1cm]{\hspace*{#1}}
%Custom functions
\DeclareMathOperator{\arccot}{arccot}